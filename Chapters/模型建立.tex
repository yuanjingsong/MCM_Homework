\section{模型建立与求解}
%对于这个问题,我们希望输入R, G, B, H, S的值,然后得到对应物质的浓度。我们知道,RGB标准和HSV标准是两套独立衡量颜色的标准,即对于一种颜色而言具有唯一的R, G, B 值和H, S, V值,并且R, G, B值和H, S, V值可以相互转换。但我们在 Adode Photoshop中,对一组RGB取值之后得到的HSV值与所给的HSV值不同,这说明我们数据中的RGB值和HSV值并不遵循我们常见的转换规则。我们常见的RGB与HSV转换方式如下:
%
%i\begin{displaymath}
%\begin{align}
%V = max(R, G, B)
%
%S = \begin{cases} 
%V - \frac{min(R, G, B)}{V} & V \neq 0 \\
%0 & otherwise
%\end{cases}
%
%H = \begin{cases}
%\frac {60(G-B)} {V-min(R, G, B)} & V \eq R \\
%120 + \frac{60(B-R)}{V-min(R,G,B)}  & V \eq G \\ 
%240 + \frac{60(R-G)}{V-min(R,G,B)} & V \eq B
%\end{cases}
%
%\begin{cases}
%H \in [0, 360\:w]
%\end{cases}
%\end{align}
%\end{displaymath}
%
%但是,根据这个公式计算得到的HSV值和我们所给的HSV值有很大的出入。所以,

\subsection{问题一的分析求解}
    \subsubsectioni{数据处理}
        因为RGB值之间具有很强的自相关性,无法对RGB三个变量建立多元回归方程,考虑到灰度处理是常见的图像处理手段,所以先将对应的RGB值转化成灰度。根据灰度计算公式
        $$HD = 0.2989R + 0.587G + 0.114B$$
        可以得到各物质在不同浓度下的灰度值。
        继续尝试计算RGB的算数平均数和几何平均数,探索浓度是否与RGB的算数平均数和几何平均数有关。
        数据处理结果如下表 :
        \begin{tabular}{|c|c|c|c|c|}
            \hline
            物质 & 浓度 & 灰度 & 算数平均数 & 几何平均数 \\
            \multirow{5}*{组胺} & 0 & 108.18 & 99 & 95.86 \\
            12.5 & 102.27 & 94.8 & 91.99 \\
            25 & 100.947 & 93.34 & 89.32 \\
            50 & 91.45 & 83.5 & 77.76 \\
            100 & 75.00 & 70.15 & 63.51 \\
            \multirow{5}*{溴酸钾} & 0 & 140.61& & \\
            12.5 & 133.35 & 123.33 & 120.11\\
            25 & 128.89 & 113.5 & 111.34 \\ 
            50 & 126.88 & 110.83 & 103.13\\
            100 & 122.22 & 95 & 51.30 \\
             \multirow{7}*{工业碱} & 0 & 140.15 & 142 & 141.76\\
             7.34 & 139.09 & 141.67 & 141.40 \\
             8.14 & 140.34 & 142 & 141.81 \\ 
             8.74 & 129.95 & 137 & 136.23 \\
             9.19 & 103.53 & 121.33 & 117.32\\
             10.18 & 62.39 & 89 & 68.21\\
             11.8 & 41.45 & 63.67 & 37.16\\
              \multirow{6}*{硫酸铝钾} & 0 & 117.64 & 114.61& 114.28\\
             0.5 & 101.87 & 105.44 & 98.76\\
             1 & 100.36 & 104.78 & 96.38\\ 
             1.5 & 98.89 & 104.95 & 93.91\\
             2 & 92.04 & 101.17 & 85.49\\
             5 & 93.45 & 100.83 & 86.75\\
               \multirow{5}*{尿中奶素} & 0 & 134.99 & 131.89 & 131.64\\
            500 & 135.48 & 131.11 & 130.69 \\
            1000 & 133.48 & 128 & 127.34\\ 
            1500 & 133.14 & 127.11 & 126.39\\
            2000 & 134.14 & 127.56 & 126.64\\
  \end{tabular}

\subsection{问题二的分析求解}
\subsection{问题三的分析}
