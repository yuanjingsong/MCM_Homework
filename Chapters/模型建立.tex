\section{模型建立与求解}
%对于这个问题,我们希望输入R, G, B, H, S的值,然后得到对应物质的浓度。我们知道,RGB标准和HSV标准是两套独立衡量颜色的标准,即对于一种颜色而言具有唯一的R, G, B 值和H, S, V值,并且R, G, B值和H, S, V值可以相互转换。但我们在 Adode Photoshop中,对一组RGB取值之后得到的HSV值与所给的HSV值不同,这说明我们数据中的RGB值和HSV值并不遵循我们常见的转换规则。我们常见的RGB与HSV转换方式如下:
%
%i\begin{displaymath}
%\begin{align}
%V = max(R, G, B)
%
%S = \begin{cases} 
%V - \frac{min(R, G, B)}{V} & V \neq 0 \\
%0 & otherwise
%\end{cases}
%
%H = \begin{cases}
%\frac {60(G-B)} {V-min(R, G, B)} & V \eq R \\
%120 + \frac{60(B-R)}{V-min(R,G,B)}  & V \eq G \\ 
%240 + \frac{60(R-G)}{V-min(R,G,B)} & V \eq B
%\end{cases}
%
%\begin{cases}
%H \in [0, 360\:w]
%\end{cases}
%\end{align}
%\end{displaymath}
%
%但是,根据这个公式计算得到的HSV值和我们所给的HSV值有很大的出入。所以,

\subsection{问题一的分析求解}
    \subsubsection{数据处理}
        因为RGB值之间具有很强的自相关性,无法对RGB三个变量建立多元回归方程,考虑到灰度处理是常见的图像处理手段,所以先将对应的RGB值转化成灰度。根据灰度计算公式
        $$HD = 0.2989R + 0.587G + 0.114B$$
        可以得到各物质在不同浓度下的灰度值。
        继续尝试计算RGB的算数平均数和几何平均数,探索浓度是否与RGB的算数平均数和几何平均数有关。
        数据处理结果如下表 :
        \begin{table}
        \begin{tabular}{|c|c|c|c|c|}
            \hline
            物质 & 浓度 & 灰度 & RGB算术平均数 & RGB几何平均数\\
            \hline
            \multirow{5}*{组胺} & 0 & 108.18 & 99 & 95.86 \\
            & 12.5 & 102.27 & 94.8 & 91.99 \\
            &25 & 100.947 & 93.34 & 89.32 \\
            &50 & 91.45 & 83.5 & 77.76 \\
            &100 & 75.00 & 70.15 & 63.51 \\
            \hline
            \multirow{5}*{溴酸钾} & 0 & 140.61&138&137.83 \\
            &12.5 & 133.35 & 123.33 & 120.11 \\
            &25 & 128.89 & 113.5 & 111.34 \\ 
            &50 & 126.88 & 110.83 & 103.13 \\
            &100 & 122.22 & 95 & 51.30 \\
            \hline
             \multirow{7}*{工业碱} & 0 & 140.15 & 142 & 141.76 \\
            & 7.34 & 139.09 & 141.67 & 141.40 \\
            & 8.14 & 140.34 & 142 & 141.81 \\ 
            & 8.74 & 129.95 & 137 & 136.23 \\
            & 9.19 & 103.53 & 121.33 & 117.32\\
            & 10.18 & 62.39 & 89 & 68.21\\
            & 11.8 & 41.45 & 63.67 & 37.16\\
            \hline
              \multirow{6}*{硫酸铝钾} & 0 & 117.64 & 114.61& 114.28 \\
            & 0.5 & 101.87 & 105.44 & 98.76\\
            & 1 & 100.36 & 104.78 & 96.38\\ 
            & 1.5 & 98.89 & 104.95 & 93.91\\
            & 2 & 92.04 & 101.17 & 85.49\\
            & 5 & 93.45 & 100.83 & 86.75\\
            \hline
               \multirow{6}*{奶中尿素} & 0 & 134.99 & 131.89 & 131.64 \\
            &500 & 135.48 & 131.11 & 130.69 \\
            &1000 & 133.48 & 128 & 127.34\\ 
            &1500 & 133.14 & 127.11 & 126.39\\
            &2000 & 134.14 & 127.56 & 126.64\\
            \hline
      \end{tabular}
  \end{table}
   根据上面的表,可以看到对于组胺和溴酸钾而言,灰度和其浓度之间有着明显的负相关,所以可以使用一元线性回归方程对其拟合,
   而工业碱,硫酸铝钾和奶中尿素对于灰度的相关性则不强,说明不能单纯使用灰度进行拟合,应该重新使用新的模型进行拟合。
  \subsubsection {组胺浓度与RGBHS之间的关系}

    将所给的数据画出对应的RGB折线图如下

    这里一张图 (RGB相关折线图)

    这里写对图的分析

    显然组胺浓度与与RGB有着强烈的负相关性,而RGB属性之间又有着自相关性,进一步计算RGB之间的自相关性系数
    \begin{table}
        \begin{tabular}{|c|c|c|c|c|c|c|}
        \hline 
            \diagbox{属性}{属性} & B & G & R & H & S & Gr \\
            \hline
            B & 1    & 0.97 & 0.87  & 0.95 & -0.99 & \null \\
            \hline
            G & 0.97 & 1    & 0.945 & 0.98 & -0.96 & \null \\
            \hline
            R & 0.87 & 0.94 &   1   & 0.89 & -0.84 & \null \\
            \hline
            H & 0.95 & 0.98 & 0.89  &   1  & -0.94 & 0.97  \\
            \hline
            S & -.99 & -0.96& -0.84 & -0.94&   1   & -0.96 \\
            \hline
        \end{tabular}
    \end{table}
    
    这里应该有一个分析图表 我不会分析

    根据所给的两组数据,我们可以看到两组数据的偏差范围不大,所以可以使用两组数据的平均数进行数据处理。分别计算R,G,B,H,S,灰度,RGB算数平均数,
    RGB几何平均数与组胺浓度的相关系数,得到对应的相关系数为
    \begin{table}
        \begin{tabular}{|c|c|c|c|c|c|c|c|c|c|c|}
        \hline         
            属性 & R & G & B & H & S & 灰度 & RGB算数平均数 & RGB几何平均数 \\
        \hline
            大小 & -0.9313 & -0.997 & -0.9724 & -0.9778 & 0.96272 & -0.996 & -0.995 & -0.993 \\
        \hline
        \end{tabular}
    \end{table}
         
    可以看到,灰度与浓度的自相关性系数绝对值较接近1,有着强烈的相关性,所以我们可以直接使用灰度作为自变量对浓度进行一元的线性回归,这样一方面简化了模型,
    另一方面并没有过多地丢失精度。

    画出组胺浓度与灰度折线图
    
    这里一张图

    这里写对图的分析

    进行一元线性回归拟合得到对应的方程为
    建立线性回归模型

    $$ y = p_1 x + p_2$$

    其中 $p_1,p_2$为回归系数。利用 Matlab 软件进行求解得到的回归系数估计值及置信区间,检验统计量 $R^2, RMSE $结果如下

    \begin{table}
        \centering
        \begin{tabular}{|c|c|c|}
            \hline
            参数     & 参数估计值  & 参数置信区间        \\ 
            \hline
            $p_1$  & -3.034 & [-3.248, -2.82] \\
            \hline
            $p_2$  & 327.4  & [306.9, 348]   \\
            \hline
            \multicolumn{3}{|c|}{$R^2$ = 0.9926 $RMSE$ = 3.404}  \\                                            
            \hline
        \end{tabular}
        \end{table}

    结果分析: 从表中可以看出 $R^2$ 近似为1,RMSE的值为3.404,且每个回归系数的置信区间没有包含零点,说明灰度值对浓度影响是显著的,
    所以模型在整体上是合理可用的。


    因为H,S与组胺的浓度的相关性系数也较大,也可以采用以H,S为自变量。进行二元函数拟合,以求得到更为精确的模型。
    二元函数拟合得到的对应方程为
    
    $$ z = p_{00} + p_{10} x + p_{01} y$$
    其中,$x$ 代表 $H$, $y$ 代表 $S$, $p_{00},p_{10} x,p_{01}$ 代表回归系数,利用 Matlab 软件进行求解得到的回归系数估计值及置信区间,
    检验统计量 $R^2, RMSE $结果如下

    \begin{table}
        \centering
        \begin{tabular}{|c|c|c|}
            \hline
            参数     & 参数估计值  & 参数置信区间        \\ 
            \hline
            $p_00$  & 60.72 & [-100.6, 222] \\
            \hline
            $p_10$  & -5.5 & [-9.439, -1.561]   \\
            \hline
            $p_01$  & 0.5614 & [-0.1248, 1.248]   \\
            \hline
            \multicolumn{3}{|c|}{$R^2$ = 0.914 $RMSE$ =7.147}  \\                                            
            \hline
        \end{tabular}
        \end{table}

    结果分析: 从表中可以看出 $R^2$ 近似为1,RMSE的值为7.147,但是注意到$p_{00}$的置信区间和$p_{01}$的置信区间含有零点,
    说明这个这个参数不是很显著,所以这个模型不能在这里使用。

    \subsubsection{溴酸钾与RGBHS之间的关系}
     
    将所给的数据画出对应的RGB折线图如下 

    这里一张图 (RGB相关折线图)    

    这里写对图的分析

    从图中可以看到,G,R所对应的两条折线趋势基本趋于平缓,大概可以判断溴酸钾浓度与G,R这两个维度没有较大的联系,进一步计算RGB之间的自相关性系数
    \begin{table}
        \begin{tabular}{|c|c|c|c|c|c|c|}
            \hline
            \diagbox{属性}{属性} & B & G & R & H & S & Gr \\
            \hline
            B & 1 & 0.89 & 0.07 & -0.85 & -0.99 & \null \\
            \hline
            G & 0.89 & 1 & 0.46 & -0.87 & -0.89 & \null \\
            \hline
            R & 0.07 & 0.46 & 1 & -0.24 & -0.05 & \null \\
            \hline
            H & -0.85 & -0.87 & -0.24 & 1 & 0.84 & -0.88 \\
            \hline
            S & -0.99 & -0.89 & -0.05 & 0.84 & 1 & -0.97 \\
            \hline
        \end{tabular}
    \end{table}
    
    这里有一个图表分析,不会

    对于所给的两组数据,两组数据的极差很小,所以可以直接使用两组数据的平均数进行数据分析,分别计算R,G,B,H,S,灰度,RGB算术平均数,
    RGB几何平均数与溴酸钾浓度的相关系数,得到如下表格

    \begin{table}
        \begin{tabular}{|c|c|c|c|c|c|c|c|c|c|c|}
            \hline         
            属性 & R & G & B & H & S & 灰度 & RGB算数平均数 & RGB几何平均数 \\
            \hline
            大小 & -0.16 & -0.87 & -0.96 & 0.69 & 0.95 & -0.95 & -0.96 & -0.96 \\
            \hline
        \end{tabular}
    \end{table}
    

    根据图表中计算的数据,可以看到灰度与浓度的相关性系数接近1,有着一定的相关性,所以我们可以直接使用灰度作为自变量对浓度进行一元的线性回归,
    一方面简化了模型的复杂度,另一方面并没有过多的丢失精度。
    
    画出溴酸钾与灰度对应的折线图进一步分析

    一张图

    这里写对图的分析

    以灰度为自变量建立线性回归模型

    $$ y = p_1 x + p_2$$

    其中 $p_1,p_2$为回归系数。利用 Matlab 软件进行求解得到的回归系数估计值及置信区间,检验统计量 $R^2, RMSE $结果如下

    \begin{table}
        \centering
        \begin{tabular}{|c|c|c|}
            \hline
            参数     & 参数估计值  & 参数置信区间        \\ 
            \hline
            $p_1$  & -5.291 & [-6.765, -3.817] \\
            \hline
            $p_2$  & 731.6 & [538, 925.2]   \\
            \hline
            \multicolumn{3}{|c|}{$R^2$ = 0.8954 $RMSE$ = 12.78}  \\                                            
            \hline
        \end{tabular}
        \end{table}
    
    从上表可以看到$R^2$近似为1,RMSE 值为12.78,并且两个参数的置信区间并没有包含零点,说明灰度值对溴酸钾浓度影响是显著的,
    因此模型在整体上是合理可用的。说明溴酸钾的浓度可以通过颜色读数来确定,其预测的方程为
    $$ y = -5.291 x + 731.6 $$ 
    其中,$x$ 为灰度值,$y$ 为溴酸钾浓度。

    根据上表可以看到RGB的几何平均数与浓度和浓度也有较强的关系,也可以尝试采用以RGB的几何平均数为自变量对浓度进行一元线性回归。
    建立一元线性回归方程为
     $$ y = p_1 x + p_2$$

    其中 $p_1,p_2$为回归系数。利用 Matlab 软件进行求解得到的回归系数估计值及置信区间,检验统计量 $R^2, RMSE $结果如下

    \begin{table}
        \centering
        \begin{tabular}{|c|c|c|}
            \hline
            参数     & 参数估计值  & 参数置信区间        \\ 
            \hline
            $p_1$  & -1.197 & [-1.367, -1.026] \\
            \hline
            $p_2$  & 162.9 & [144.3, 181.4]   \\
            \hline
            \multicolumn{3}{|c|}{$R^2$ = 0.9704 $RMSE$ = 6.8}  \\                                            
            \hline
        \end{tabular}
        \end{table}

    从上表可以看到$R^2$ 近似值为1,RMSE值为6.8,并且两个参数的置信区间并没有包含零点,说明RGB的几何平均数对溴酸钾浓度影响是显著的,
    因此模型在整体上是合理可用的。说明溴酸钾的浓度可以通过颜色读数来确定,其预测的方程为
    $$ y = -1.197 x + 162.9 $$

    其中,$x$ 为RGB的几何平均数,$y$ 为溴酸钾浓度。

    同时根据上面的图,发现H,S与溴酸钾浓度的相关系数也很大,所以以H,S为自变量进行了多元函数的拟合分析,
    建立二元回归方程 为

    $$ z = p_{00} + p_{10} x + p_{01} y$$
    其中,$x$ 代表 $H$, $y$ 代表 $S$, $p_{00},p_{10} x,p_{01}$ 代表回归系数,利用 Matlab 软件进行求解得到的回归系数估计值及置信区间,
    检验统计量 $R^2, RMSE $结果如下

    \begin{table}
        \centering
        \begin{tabular}{|c|c|c|}
            \hline
            参数     & 参数估计值  & 参数置信区间        \\ 
            \hline
            $p_00$  & 155.8 & [-29.68, 341.2] \\
            \hline
            $p_10$  & -7.897 & [-15.92, 0.1276]   \\
            \hline
            $p_01$  & 0.6479 & [0.4536, 0.8423]   \\
            \hline
            \multicolumn{3}{|c|}{$R^2$ = 0.9478 $RMSE$ =9.653}  \\                                            
            \hline
        \end{tabular}
        \end{table}

    结果分析: 从表中可以看出 $R^2$ 近似为1,RMSE的值为9.653,但是注意到$p_{00}$的置信区间和$p_{01}$的置信区间含有零点,
    说明这个这个参数不是很显著,所以这个模型不能在这里使用。

    \subsubsection{工业碱浓度与RGBHS的关系}
    将所给的数据画出对应的RGB折线图如下

    这里一张图(RGB相关折线图)

    这里写对图的分析
    
    从图中可以看到,浓度为0的点和浓度为7.34的点所对应的RGBHS值大体相等,但由于其浓度差较大,所以判断浓度为0的点不是一个有效数据,
    故排除浓度为0的点进行分析。

    注意到图中的 RGB 三条曲线所对应的趋势相同,变化不大,进一步计算他们之间的相关性系数
    \begin{table}
            \begin{tabular}{|c|c|c|c|c|c|c|}
                \hline
                \diagbox{属性}{属性} & B & G & R & H & S & Gr \\
                \hline
                B & 1 & 0.83 & 0.86 & -0.71 & -0.83 & \null \\
                \hline
                G & 0.83 & 1 & 0.87 & -0.97 & -0.99 & \null \\
                \hline
                R & 0.86 & 0.87 & 1 & -0.84 & -0.88 & \null \\
                \hline
                H & -0.71 & -0.97 & -0.84 & 1 & 0.97 & -0.96 \\
                \hline
                S & -0.83 & -0.99 & -0.88 & 0.97 & 1 & -0.99 \\
                \hline
            \end{tabular}
        \end{table}
   
    这里有一个图表分析,不会

    画出工业碱与灰度的折线图如下

    这里一张图

    这里分析一下图

    所以可以使用灰度进行一元回归分析
    建立线性回归模型
    $$ y = p_1 x + p_2 $$

    其中,$p_1, p_2$为回归系数。利用 Matlab 软件进行求解得到的回归系数估计值及置信区间,
    检验统计量$R^2, RMSE $结果如下 

    \begin{table}
        \centering
        \begin{tabular}{|c|c|c|}
            \hline
            参数     & 参数估计值  & 参数置信区间        \\ 
            \hline
            $p_1$  & -0.03599 & [-0.05097, -0.02101] \\
            \hline
            $p_2$  & 12.93  & [11.29, 14.58]   \\
            \hline
            \multicolumn{3}{|c|}{$R^2$ = 0.9175 $RMSE$ = 0.5079}  \\                                            
            \hline
        \end{tabular}
        \end{table}

    结果分析: 从表中可以看出 $R^2$ 近似为1,RMSE的值为0.5079,且每个回归系数的置信区间没有包含零点,
    说明灰度值对浓度影响是显著的,所以模型在整体上是合理可用的。

     
\subsection{问题二的分析求解}

\subsection{问题三的分析}
