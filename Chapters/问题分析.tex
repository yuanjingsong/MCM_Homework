\section{问题分析}
\subsection{问题一分析}
问题为对于某种物质,输入其不同的颜色维度,得到对应的浓度值。Data1中的数据分别给了物质名称,物质浓度,R, G, B, H, S 这几个维度的量,颜色读数有两套独立的规则体系,分别是RGB体系与HSV体系,两套体系之间可以互相转换。所以,我们先利用主成分分析法,找出在不同物质中,各个属性值之间的相关性,对于相关性大的,我们可以采取降维处理的方式,减少变量个数,简化模型,对于相关性小的,我们则必须利用不同的属性,建立不同的模型,然后验证模型的拟合结果,给出实验数据优劣的评价标准,并评价五组数据的优劣。

\subsection{问题二分析}

\subsection{问题三分析}
