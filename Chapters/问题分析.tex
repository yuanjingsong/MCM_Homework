\section{问题分析}
\subsection{问题一分析}
问题一是对于某种物质,输入其不同的颜色维度,得到对应的浓度值。附件1中的数据分别给了物质名称,
物质浓度,R, G, B, H, S 这几个维度的量,颜色读数有两套独立的规则体系,分别是RGB体系与HSV体系
,两套体系之间可以互相转换。所以,我们先利用主成分分析法,找出对于不同物质各个属性值之间的相关性,
对于相关性大的,我们可以采取降维处理的方式,减少变量个数,简化模型,对于相关性小的,我们则必须利用不同的属性,
建立不同的模型。对于建立的模型,我们采用 $R^2$, $RMSE$两个指标去衡量模型优劣,同时采用参数的置信区间
筛选参数,选择模型是否使用。
对于数据评价标准,我们建立了衡量数据评价的模型,利用数据评价模型,我们评价不同物质的数据质量。

\subsection{问题二分析}
问题二是对于附件二所给的数据建立合适的模型。对于附件二所给的数据,经过计算,发现所给的数据有一定的误差,
进行数据处理之后,建立灰度与浓度的一元线性回归模型。发现模型效果不好后,考虑建立非线性模型,尝试建立指数增长模型
和快速增长模型,同样采用$R^2$,$RMSE$这两个指标评价模型建立的效果,选取适当的模型。

\subsection{问题三分析}
问题三是对于不同的数据量和颜色维度对数学模型的影响。所以我们选择了数据量过少的溴酸钾与其他物质进行
对比,比较数据量的多少会对数学模型的精度产生的影响。同时,对于同一个物质,我们比较了不同维度下建立的
数学模型之间准确度,从而比较维度是否会对建立的数学模型精度产生影响。
