%讨论数据量和颜色维度对模型的影响
\section{问题三的分析求解}
\subsection{数据量对模型的影响}
   本文采取的大部分模型是回归分析模型,对样本数据量有很大的要求,样本容量过少,范围过小会导致出现拟合偏差或者过拟合的现象,
   也很难保证参数的精确度。因此,对于一个模型的建立,样本容量应该越多越好,但同时,样本数据不应该仅注重浓度范围的大小,还应该注重
   同一个浓度下数据的重复,单一一组数据无法说明在该浓度下数据的准确性,对于我们选择数据造成了影响。
   
   例如问题中的溴酸钾浓度,其数据范围 不仅过于集中,同时数据量过少,不可以进行数据间的交叉验证,
   甚至出现了一些不可用的数据,这极大地影响了我们筛选数据。对其所建立的模型 就具有一定的局限性,且无法保证其精确度。

   而硫酸铝钾所对应的数据量较为充足,可以进行数据之间的交叉验证,方便我们选择数据,数据的范围也比较大,提高了我们建立
   模型的准确度和使用范围。所以我们对其建立的模型就具有更多的普适性。
   
   但在建立模型中,数据量也并非越多越好,过大的数据量可能会造成数据处理的困难,几百上千万的
   数据量对我们筛选数据造成了影响。过多的数据量对于数据收集也是一个挑战。
   总之,在建立数学模型中,我们选择的数据必须具有代表性,可以代表多个情况下的均值,同时所选取的数据范围也应该比较宽泛,可以更好的
   反应不同范围下的数据变化情况。

\subsection{颜色维度对模型的影响}
    题目中所给的数据维度分别是 R, G, B, H 和 S 而在现有的颜色体系中,RGB体系和HSV体系是等价的,所以,所给的五个维度可以转化为
    3个维度, 但是在上文中已经分析,所给的HS两个维度由于没有给出定义域的原因,是需要重新计算的。我们知道灰度和RGB之间可以建立一个
    函数关系$Gr = f(R, G, B)$,每组RGB值仅有唯一的灰度与之对应,但同一个灰度可能有多组RGB对与之对应。显然这个函数不是一个单射
    函数,所以对于我们所建立的灰度与物质浓度的函数,我们总可以用RGB表示,但是,如果我们使用RGB三个维度进行多元函数拟合,则不一定可以
    降维。所以我们可以使用使用三变量的多元函数拟合也可是使用灰度模型进行一元回归拟合。
    
    综上,我们可以直接使用题目所给的5个维度进行多元函数拟合,也可以使用RGB体系进行拟合,降低维度,也可以使用灰度模型,进一步降低
    维度。选取过多的变量不一定能带来较好的拟合结果,所以并非是维度越多越好,而是我们所选的变量能否准确描述浓度的变化。
    而在数学建模中,模型所使用的变量越多往往意味着计算量大,所以,如果更少的变量建立的模型可以使用,就没有必要使用更多变量建立的模型了。