\begin{abstract}

为了更精确地通过溶液的颜色读数确定物质的浓度,本文研究了题目提供的各物质颜色读数与物质浓度的关系。

对于问题一各物质,首先对原始数据进行处理,引入灰度值为数据降维,并计算各参数之间的相关系数与各参数对于物质浓度的相关系数。
使用灰度值分别对组胺浓度和溴酸钾浓度进行回归分析,得到结果为$\text{组胺浓度}=-3.034\times \text{灰度}+327.4$,
$\text{溴酸钾浓度}=-5.291\times \text{灰度}+731.6$。
对于工业碱,直接使用灰度进行回归分析效果不显著,去除掉数据中的离群点(浓度为0的点)后,回归效果显著,但只适用于浓度在$[7,12]$区间的情况,
结果为$\text{工业碱浓度}=-0.03599\times \text{灰度}+12.93$。
对于硫酸铝钾,首先通过分析其数据特征建立通过RGB平均数数值判定其浓度是否小于0.5ppm的判定模型。
由于H、S随浓度的变化符合酶促反应速率特点,引入描述酶促反应的指数增长模型和快速平衡模型,
得到的最优结果为$\text{硫酸铝钾浓度}=(0.38H-28.35)\div(510-5H)$。
对于奶中尿素,使用参数B进行一维线性回归,得到结果为$\text{奶中尿素浓度}=-129.7B+ 15490$。

设计准确度、稳定度、区分度、吻合度四个指标,建立综合评价模型以进行数据质量评价。
其中吻合度为HS测量值与计算值的相差程度,由于吻合度的区分度太小,故将其从模型中舍去。
准确度用所测数据的组数来衡量,稳定度用同物质同浓度下数据变异系数来衡量,区分度用同物质异浓度的数据离散度来衡量。
最终得出的数据质量评价得分为:组胺0.4236,溴酸钾0.5538,工业碱0.4977,硫酸铝钾0.5014,奶中尿素0.0888。

对于问题二,通过R、G、B计算出H、S的计算值与测量值对比,发现数据表中H、S两列位置恰好相反,故将其修正。
先利用数据质量的综合评价模型检验其数据质量,发现质量良好,可用于分析。
绘制折线图,发现灰度和参数G与浓度有较为明显的线性关系,尝试建立一元线性回归模型,结果效果不佳。
使用指数增长模型对灰度与浓度进行回归分析,得到回归系数的置信区间包含零点,无法采纳。
再使用快速平衡模型对参数S和浓度进行分析,得到的结果较好,可以适用于实际:$\text{二氧化硫浓度}=(124.1S-359.45)\div(142.01-S)$。

对于问题三,分别说明数据量和数据维度对于模型质量的影响。
通过对比数据量较少的溴酸钾和数据量较多的硫酸铝钾,得出数据量越大模型的精度和普适性越高,但也会加大数据处理难度的结论。
通过讨论RGB模型和HSV模型的转换关系,对比问题一、二中各项回归分析,
得出结论为:数据维度多则包含信息越多,但不一定能得到更准确的模型,在使用维度较低的数据建成的模型可用的情况下,则没有必要对更多维的数据进行讨论。

\keywords{回归分析\quad  酶促反应\quad  数码照片比色法\quad  灰度\quad   RGB和HSV\quad }
\end{abstract}