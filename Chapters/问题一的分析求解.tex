\section{问题一的分析求解}
\subsection{数据处理}
        因为RGB值之间具有很强的自相关性,无法对RGB三个变量建立多元回归方程,考虑到灰度$^{\cite{gr}}$处理是常见的图像处理手段,
        所以先将对应的RGB值转化成灰度。根据灰度计算公式
        $$Gr = 0.2989R + 0.587G + 0.114B$$
        可以得到各物质在不同浓度下的灰度值。
        继续尝试计算RGB的算数平均数和几何平均数,探索浓度是否与RGB的算术平均数和几何平均数有关。
  %      \begin{table}
  %      \begin{tabular}{|c|c|c|c|c|}
  %          \hline
  %          物质 & 浓度 & 灰度 & RGB算术平均数 & RGB几何平均数\\
  %          \hline
  %          \multirow{5}*{组胺} & 0 & 108.18 & 99 & 95.86 \\
  %          & 12.5 & 102.27 & 94.8 & 91.99 \\
  %          &25 & 100.947 & 93.34 & 89.32 \\
  %          &50 & 91.45 & 83.5 & 77.76 \\
  %          &100 & 75.00 & 70.15 & 63.51 \\
  %          \hline
  %          \multirow{5}*{溴酸钾} & 0 & 140.61&138&137.83 \\
  %          &12.5 & 133.35 & 123.33 & 120.11 \\
  %          &25 & 128.89 & 113.5 & 111.34 \\ 
  %          &50 & 126.88 & 110.83 & 103.13 \\
  %          &100 & 122.22 & 95 & 51.30 \\
  %          \hline
  %           \multirow{7}*{工业碱} & 0 & 140.15 & 142 & 141.76 \\
  %          & 7.34 & 139.09 & 141.67 & 141.40 \\
  %          & 8.14 & 140.34 & 142 & 141.81 \\ 
  %          & 8.74 & 129.95 & 137 & 136.23 \\
  %          & 9.19 & 103.53 & 121.33 & 117.32\\
  %          & 10.18 & 62.39 & 89 & 68.21\\
  %          & 11.8 & 41.45 & 63.67 & 37.16\\
  %          \hline
  %            \multirow{6}*{硫酸铝钾} & 0 & 117.64 & 114.61& 114.28 \\
  %          & 0.5 & 101.87 & 105.44 & 98.76\\
  %          & 1 & 100.36 & 104.78 & 96.38\\ 
  %          & 1.5 & 98.89 & 104.95 & 93.91\\
  %          & 2 & 92.04 & 101.17 & 85.49\\
  %          & 5 & 93.45 & 100.83 & 86.75\\
  %          \hline
  %             \multirow{6}*{奶中尿素} & 0 & 134.99 & 131.89 & 131.64 \\
  %          &500 & 135.48 & 131.11 & 130.69 \\
  %          &1000 & 133.48 & 128 & 127.34\\ 
  %          &1500 & 133.14 & 127.11 & 126.39\\
  %          &2000 & 134.14 & 127.56 & 126.64\\
  %          \hline
  %    \end{tabular}
  %\end{table}
   根据数据处理结果,可以看到对于组胺和溴酸钾而言,灰度和其浓度之间有着明显的负相关,所以可以使用一元线性回归方程对其拟合,
   而工业碱,硫酸铝钾和奶中尿素对于灰度的相关性则不强,说明不能单纯使用灰度进行拟合,应该重新使用新的模型进行拟合。
\subsection {对组胺浓度与颜色读数间关系的分析}

    将所给的数据画出对应的RGB折线图如图~\ref{ZA_RGB_C}所示。

    \quickF{0.6}{ZA_RGB_C.png}{组胺浓度关于RGB的折线图}{ZA_RGB_C}

    根据图~\ref{ZA_RGB_C},可以看到R,G,B均与组胺浓度$c$有着明显的线性关系,且随着组胺浓度增加,均有着
    减少的趋势,并且三者的变化范围较大,可以看出较大的差距。

    显然组胺浓度与与RGB有着强烈的负相关性,而RGB属性之间又有着自相关性,进一步计算RGB之间的自相关性系数
    如表~\ref{组胺各颜色参数相关系数}所示。
      \begin{table}[H]
        \caption{组胺各颜色参数间的自相关系数}
        \label{组胺各属性相关系数}
        \centering
        \begin{tabular}{|c|c|c|c|c|c|c|}
        \hline
            \diagbox{颜色参数}{颜色参数} & B & G & R & H & S & Gr \\
            \hline
            B & 1    & 0.97 & 0.87  & 0.95 & -0.99 & \null \\
            \hline
            G & 0.97 & 1    & 0.945 & 0.98 & -0.96 & \null \\
            \hline
            R & 0.87 & 0.94 &   1   & 0.89 & -0.84 & \null \\
            \hline
            H & 0.95 & 0.98 & 0.89  &   1  & -0.94 & 0.97  \\
            \hline
            S & -.99 & -0.96& -0.84 & -0.94&   1   & -0.96 \\
            \hline
        \end{tabular}
    \end{table}

    从表~\ref{组胺各属性相关系数}可以看出,B,G,R均有着较大的相关性,所以如果直接使用三元变量进行拟合
    可能不会得到较好的结果,所以可以使用灰度作为自变量,将每组RGB值转化为灰度,进行一元线性拟合。

    根据所给的两组数据,我们可以看到两组数据的偏差范围不大,所以可以使用两组数据的平均数进行数据处理。
    进一步计算R,G,B,H,S,Gr,RGB算数平均数,RGB几何平均数与组胺浓度的相关系数,得到对应的相关系数如表~\ref{ZuAnCov}所示。
    \begin{table}[H]
        \centering
        \caption{各颜色参数与组胺浓度的相关系数}
        \label{ZuAnCov}
        \begin{tabular}{@{}ccccccccc@{}}
        \toprule
        颜色参数 & R       & G      & B       & H       & S       &  Gr    & RGB算术平均数 & RGB几何平均数 \\ \midrule
        相关系数 & -0.9313 & -0.997 & -0.9724 & -0.9778 & 0.96272 & -0.996 & -0.995   & -0.993   \\ \bottomrule
        \end{tabular}
        \end{table}
         
    由表~\ref{ZuAnCov}可以看到,灰度与组胺浓度的相关性系数绝对值较接近1,有着强烈的相关性,所以我们
    可以直接使用灰度作为自变量对浓度进行一元线性回归,这样一方面简化了模型,
    另一方面并没有过多地丢失精度。

    绘制组胺浓度与灰度折线图如图~\ref{ZA_Gr_C}所示。

    \quickF{0.6}{ZA_Gr_C.png}{组胺浓度关于灰度的折线图}{ZA_Gr_C}

    根据所绘制的图~\ref{ZA_Gr_C},可以看出灰度与组胺浓度近似服从一元线性模型,建立线性回归模型
    $$ y = p_1 \times x + p_2$$
    其中 $p_{1}, p_{2}$为回归系数,$x$为灰度,$y$为组胺浓度。利用 Matlab 软件进行求解得到的回归系数估计值及置信区间,检验统计量 $R^2, RMSE $结果如表~\ref{ZuAnLinear}所示。

    \begin{table}[H]
        \centering
        \caption{一元拟合的参数结果}
        \label{ZuAnLinear}
        \begin{tabular}{@{}ccc@{}}
        \toprule
        参数         & 参数估计值      & 参数置信区间                  \\ \midrule
        $p_1$      & -3.034     & {[}-3.248, -2.82{]}     \\
        $p_2$      & 327.4      & {[}306.9, 348{]}     \\
        \hline
        \multicolumn{3}{c}{$R^2$ = 0.9926 $RMSE$ = 3.404} \\ \bottomrule
        \end{tabular}
        \end{table}

    结果分析: 从表~\ref{ZuAnLinear}可以看出 $R^2$ 近似为1,$RMSE$的值为3.404,且每个回归系数的置信区间没有包含零点,说明灰度值对浓度影响是显著的,
    所以模型在整体上是合理可用的。说明组胺的浓度可以通过颜色读数来确定,其预测的方程为
    $$y = -3.034Gr + 327.4 $$
    
    因为H,S与组胺的浓度的相关性系数也较大,也可以尝试用以H,S为自变量。进行二元函数线性回归拟合,以求得到更为精确的模型。
    建立二元函数拟合方程为
    $$ z = p_{00} + p_{10} \times x + p_{01} \times y$$
    其中,$x$ 为 H, $y$ 为 S,$z$为组胺浓度,$p_{00},p_{10},p_{01}$ 为回归系数,利用 Matlab 软件进行求解得到回归系数估计值及置信区间,
    检验统计量 $R^2, RMSE $结果如表~\ref{ZuAn2Dim}所示。

    \begin{table}[H]
        \centering
        \caption{二元拟合的参数结果}
        \label{ZuAn2Dim}
        \begin{tabular}{@{}lcc@{}}
        \toprule
        参数       & 参数估计值  & 参数置信区间                       \\ \midrule
        $p_{00}$ & 60.72  & {[}-100.6, 222{]}    \\
        $p_{10}$ & -5.5   & {[}-9.439, -1.561{]} \\
        $p_{01}$ & 0.5614 & {[}-0.1248, 1.248{]} \\
        \hline
        \multicolumn{3}{c}{$R^2$ = 0.914 $RMSE$ =7.147}  \\ \bottomrule
        \end{tabular}
        \end{table}

    结果分析: 从表~\ref{ZuAn2Dim}中可以看出 $R^2$ 近似为1,$RMSE$的值为7.147,但是注意到$p_{00}$的置信区间和$p_{01}$的置信区间含有零点,
    说明这两个参数不是很显著,所以这个模型不在这里适用。

\subsection{对溴酸钾浓度与颜色读数间关系的分析}

    将所给的数据画出对应的RGB折线图,如图~\ref{XSJ_RGB_C}所示。

    \quickF{0.6}{XSJ_RGB_C.png}{溴酸钾浓度关于RGB的折线图}{XSJ_RGB_C}

    从图中可以看到,随着溴酸钾浓度的增加,G,R所对应的两条折线趋势基本趋于平缓, 大概可以判断溴酸钾浓度与G,R这两个维度没有较大的联系,
    进一步计算RGB之间的自相关性系数如表~\ref{溴酸钾相关性系数图}所示。
    \begin{table}[H]
        \centering
        \caption{溴酸钾各颜色参数的自相关性系数}
        \label{溴酸钾相关性系数图}
        \begin{tabular}{|c|c|c|c|c|c|c|}
            \hline
            \diagbox{颜色参数}{颜色参数} & B & G & R & H & S & Gr \\
            \hline
            B & 1 & 0.89 & 0.07 & -0.85 & -0.99 & \null \\
            \hline
            G & 0.89 & 1 & 0.46 & -0.87 & -0.89 & \null \\
            \hline
            R & 0.07 & 0.46 & 1 & -0.24 & -0.05 & \null \\
            \hline
            H & -0.85 & -0.87 & -0.24 & 1 & 0.84 & -0.88 \\
            \hline
            S & -0.99 & -0.89 & -0.05 & 0.84 & 1 & -0.97 \\
            \hline
        \end{tabular}
    \end{table}

    根据图~\ref{溴酸钾相关性系数图}可以看到,R与其他变量相关性系数较差,
    但B与G相关性系数较强,所以不能直接使用三元变量进行多元函数拟合,可以
    考虑使用灰度进行一元线性回归拟合。

    对于所给的两组数据,两组数据的极差很小,所以可以直接使用两组数据的平均数进行数据分析,分别计算R,G,B,H,S,灰度,RGB算术平均数,
    RGB几何平均数与溴酸钾浓度的相关系数,得到对应的相关系数如表~\ref{多变量与溴酸钾浓度}所示。

    \begin{table}[H]
        \centering
        \caption{各颜色参数与溴酸钾浓度的相关系数}
        \label{多变量与溴酸钾浓度}
        \begin{tabular}{@{}ccccccccc@{}}
        \toprule
        参数 & R     & G     & B     & H    & S    & 灰度    & RGB算数平均数 & RGB几何平均数 \\ \midrule
        相关系数 & -0.16 & -0.87 & -0.96 & 0.69 & 0.95 & -0.95 & -0.96    & -0.96    \\ \bottomrule
        \end{tabular}
        \end{table}

    由表~\ref{多变量与溴酸钾浓度}中计算的数据,可以看到灰度与浓度的相关性系数绝对值接近1,有着一定的相关性,
    所以我们可以直接使用灰度作为自变量对溴酸钾浓度进行一元的线性回归。这样,一方面简化了模型的复杂度,另一方面并没有过多的丢失精度。

    绘制溴酸钾浓度与灰度折线图如图~\ref{XSJ_Gr_C}所示。

    \quickF{0.6}{XSJ_Gr_C.png}{溴酸钾浓度关于灰度的折线图}{XSJ_Gr_C}

    从图\ref{XSJ_Gr_C}可以看出,灰度与溴酸钾浓度点近似服从线性分布。 以灰度为自变量建立线性回归模型
    $$ y = p_1 \times x + p_2$$
    其中 $p_{1},p_{2}$为回归系数, $x$为灰度,$y$为溴酸钾浓度。利用 Matlab 软件进行求解得到回归系数估计值及置信区间,
    检验统计量 $R^2, RMSE $结果如表~\ref{溴酸钾一元拟合}所示。

    \begin{table}[H]
        \centering
        \caption{以灰度为自变量一元拟合结果}
        \label{溴酸钾一元拟合}
        \begin{tabular}{@{}ccc@{}}
        \toprule
        参数        & 参数估计值      & 参数置信区间                   \\ \midrule
        $p_1$     & -5.291     & {[}-6.765, -3.817{]}     \\
        $p_2$     & 731.6      & {[}538, 925.2{]}         \\
        \hline
        \multicolumn{3}{c}{$R^2$ = 0.8954 $RMSE$ = 12.78} \\ \bottomrule
        \end{tabular}
        \end{table}
    从表~\ref{溴酸钾一元拟合}可以看到$R^2$近似为1,$RMSE$值为12.78,并且两个参数的置信区间并没有包含零点,说明灰度值对溴酸钾浓度影响是显著的,
    因此模型在整体上是合理可用的。说明溴酸钾的浓度可以通过颜色读数来确定,其预测的方程为
    $$ y = -5.291Gr + 731.6 $$ 

    根据表\ref{多变量与溴酸钾浓度}可以看到RGB的几何平均数与溴酸钾浓度也有较强的关系,也可以尝试采用以RGB的几何平均数为自变量对浓度进行一元线性回归。
    建立一元线性回归方程为
     $$ y = p_1 \times x + p_2$$
    其中 $x$为RGB几何平均数,$y$为溴酸钾浓度,$p_1, p_2$为回归系数。利用 Matlab 软件进行求解得到的回归系数估计值及置信区间,检验统计量 $R^2, RMSE $结果如表~\ref{RGB拟合}

    \begin{table}[H]
        \centering
        \caption{以RGB几何平均数为自变量一元拟合结果}
        \label{RGB拟合}
        \begin{tabular}{@{}ccc@{}}
        \toprule
        参数        & 参数估计值     & 参数置信区间                  \\ \midrule
        $p_1$     & -1.197    & {[}-1.367, -1.026{]}    \\
        $p_2$     & 162.9     & {[}144.3, 181.4{]}      \\
        \hline
        \multicolumn{3}{c}{$R^2$ = 0.9704 $RMSE$ = 6.8} \\ \bottomrule
        \end{tabular}
        \end{table}

    从表~\ref{RGB拟合}可以看到$R^2$ 近似值为1,$RMSE$值为6.8,并且两个参数的置信区间并没有包含零点,说明RGB的几何平均数对溴酸钾浓度影响是显著的,
    因此模型在整体上是合理可用的。说明溴酸钾的浓度可以通过颜色读数来确定,其预测的方程为
    $$ y = -1.197 x + 162.9 $$

    同时根据上面的表\ref{溴酸钾相关性系数图},发现H,S与溴酸钾浓度的相关系数也很大,所以以H,S为自变量进行了多元函数的拟合分析,
    建立二元回归方程 为
    $$ z = p_{00} + p_{10} \times x + p_{01} \times y$$
    其中,$x$ 为$H$, $y$ 为 $S$,$z$为溴酸钾浓度,$p_{00},p_{10},p_{01}$ 代表回归系数,利用 Matlab 软件进行求解得到的回归系数估计值及置信区间,
    检验统计量 $R^2, RMSE $结果如表~\ref{二元拟合结果}所示。

    \begin{table}[H]
        \centering
        \caption{以H,S为自变量二元拟合结果}
        \label{二元拟合结果}
        \begin{tabular}{@{}ccc@{}}
        \toprule
        参数         & 参数估计值     & 参数置信区间                  \\ \midrule
        $p_{00}$     & 155.8     & {[}-29.68, 341.2{]}     \\
        $p_{10}$     & -7.897    & {[}-15.92, 0.1276{]}    \\
        $p_{01}$     & 0.6479    & {[}0.4536, 0.8423{]}    \\
        \hline
        \multicolumn{3}{c}{$R^2$ = 0.9478 $RMSE$ =9.653} \\ \bottomrule
        \end{tabular}
        \end{table}

    结果分析: 从表~\ref{二元拟合结果}可以看出 $R^2$ 近似为1,$RMSE$的值为9.653,但是注意到$p_{00}$的置信区间和$p_{01}$的置信区间含有零点,
    说明这个这个参数不是很显著,所以这个模型不能在这里使用。

\subsection{对工业碱浓度与颜色读数间关系的分析}
    通过分析原始数据,发现浓度为0的点和浓度为7.34的点所对应的RGBHS值大体相等,但由于浓度为0的点与其他浓度测量值偏离较大,所以可以判断浓度为0的点不是一个有效数据,
    故排除浓度为0的点进行分析。将浓度为0的点剔除后,绘制对应的RGB折线图如图~\ref{GYJ_RGB_C_no0}所示。

    \quickF{0.6}{GYJ_RGB_C_no0.png}{工业碱浓度去除零点后关于RGB的折线图}{GYJ_RGB_C_no0}

    由图~\ref{GYJ_RGB_C_no0}可以发现,RGB 三条曲线所对应的趋势基本相同相同,变化不大,
    判断RGB之间可能有较大的线性相关性。进一步计算他们之间的相关性系数如表~\ref{工业碱浓度相关系数图}所示。

    \begin{table}[H]
        \centering
        \label{工业碱浓度相关系数图}
        \caption{工业碱各个属性间的自相关系数}
            \begin{tabular}{|c|c|c|c|c|c|c|}
                \hline
                \diagbox{属性}{属性} & B & G & R & H & S & Gr \\
                \hline
                B & 1 & 0.83 & 0.86 & -0.71 & -0.83 & \null \\
                \hline
                G & 0.83 & 1 & 0.87 & -0.97 & -0.99 & \null \\
                \hline
                R & 0.86 & 0.87 & 1 & -0.84 & -0.88 & \null \\
                \hline
                H & -0.71 & -0.97 & -0.84 & 1 & 0.97 & -0.96 \\
                \hline
                S & -0.83 & -0.99 & -0.88 & 0.97 & 1 & -0.99 \\
                \hline
            \end{tabular}
        \end{table}
   
    根据表~\ref{工业碱浓度相关系数图}可以看到 RGB 之间相关性系数较大,
    所以不能直接使用三元变量进行线性回归拟合,可以尝试使用灰度进行一元线性回归拟合。

    绘制工业碱浓度与灰度折线图如图~\ref{GYJ_Gr_C}所示。

    \quickF{0.6}{GYJ_Gr_C.png}{工业碱浓度关于灰度的折线图}{GYJ_Gr_C}

    从图~\ref{GYJ_Gr_C}可以看到,在去除掉浓度为0的点后,工业碱浓度与灰度近似服从一元线性模型,
    所以可以使用灰度进行一元回归分析,建立一元线性回归模型
    $$ y = p_1 \times x + p_2 $$
    其中,$x$为灰度,$y$为工业碱浓度,$p_1, p_2$为回归系数。利用 Matlab 软件进行求解得到的回归系数估计值及置信区间,
    检验统计量$R^2, RMSE $结果如下 

    \begin{table}[H]
        \centering
        \caption{以灰度为自变量一元拟合结果}
        \label{灰度工业碱一元拟合}
        \begin{tabular}{@{}ccc@{}}
        \toprule
        参数       & 参数估计值      & 参数置信区间                     \\ \midrule
        $p_1$    & -0.03599   & {[}-0.05097, -0.02101{]}   \\
        $p_2$    & 12.93      & {[}11.29, 14.58{]}         \\
        \hline
        \multicolumn{3}{c}{$R^2$ = 0.9175 $RMSE$ = 0.5079} \\ \bottomrule
        \end{tabular}
        \end{table}

    结果分析: 从表~\ref{灰度工业碱一元拟合}可以看出 $R^2$ 近似为1,$RMSE$的值为0.5079,且每个回归系数的置信区间没有包含零点,
    说明灰度值对浓度影响是显著的,所以模型在整体上是合理可用的。说明工业碱溶液的浓度可以通过颜色读数来确定,其预测方程为:
        $$y = -0.03599 x + 12.93 $$
    注意:由于所给的数据量过少,且浓度变化范围主要集中在 $7 \sim 12$之间,所以我们根据灰度来预测时灰度的取值范围也应在 $40 \sim 140$
    之间。
\subsection{硫酸铝钾颜色与浓度的关系}

    将硫酸铝钾的R,G,B,H,S数据按浓度分组,并计算每组平均值,RGB算术平均数和几何平均数。
    取计算所得平均值绘制 R, G, B,灰度与浓度的折线图如图~\ref{LSLJ_RGBGr_C}所示。

    \quickF{0.6}{LSLJ_RGBGr_C.png}{硫酸铝钾浓度关于RGB和灰度的折线图}{LSLJ_RGBGr_C}
    
    观察图~\ref{LSLJ_RGBGr_C}发现,随着硫酸铝钾浓度的增加, R, G, B的值没有明显的变化趋势,只在浓度为0和浓度为0.5$ppm$及以上有明显的差异。因此分为保留浓度为0的数据和除去浓度为0的数据两种情况分别分析。
    计算灰度和浓度的相关性如表~\ref{硫酸铝钾浓度和灰度}所示。
    \begin{table}[H]
      \centering
      \caption{灰度和硫酸铝钾浓度的相关性}
      \label{硫酸铝钾浓度和灰度}
      \begin{tabular}{@{}ccc@{}}
      \toprule
      浓度 & 包括0     & 不包括0    \\ \midrule
      大小 & -0.6778 & -0.7195 \\ \bottomrule
      \end{tabular}
    \end{table}
    
    观察表~\ref{硫酸铝钾浓度和灰度}发现灰度和浓度的相关性不强。无法建立有关模型。
    
    而由于浓度为0的数据和浓度为0.5ppm及以上的数据之间有较为明显的差异,若只需要粗略估计浓度,可以建立一个简略的颜色与浓度的模型。
    $$ c<0.5 \text{ ppm}, \text{if } aver(RGB)>105 $$
    式中 R, G, B的取值范围均在0到255之间,$aver(RGB)$为RGB三个数的数值的算术平均数。
    
    绘制H, S和浓度的折线图如图~\ref{LSLJ_HS_C}所示。

    \quickF{0.6}{LSLJ_HS_C.png}{硫酸铝钾浓度关于HS的折线图}{LSLJ_HS_C}
    
    发现随着浓度的增加,H, S两类数据的变化规律类似,均为在浓度低是快速增加,而浓度高时缓慢增加。
    由此联想到酶促反应$^{\cite{mm}}$$^{\cite{exp}}$中反应速度和底物浓度之间的关系。对于酶促反应的性质,满足的模型有两个:
    
    指数增长模型:$$y=\beta_0(1-e^{-\beta_1x})$$

    $Michaelis-Menten$模型(米-曼式模型,也叫快速平衡模型):
    $$y=\frac{\beta_0x}{\beta_1+x}$$
    
    首先用指数增长模型对该组数据进行回归分析,其中 H 值和 S 值作为自变量,硫酸铝钾浓度作为因变量,得到结果:
    
    \begin{table}[H]
      \centering
      \caption{硫酸铝钾浓度和H值指数增长模型回归分析结果}
      \label{硫酸铝钾浓度和H值指数}
      \begin{tabular}{@{}ccc@{}}
      \toprule
      参数        & 参数估计值      & 参数置信区间                   \\ \midrule
      $\beta_0$     & -3.065e-12     & {[}-2.316e-10, 2.255e-10{]}     \\
      $\beta_1$     & -0.2703   & {[}-1.006, 0.4653{]}    \\
      \hline
      \multicolumn{3}{c}{$R^2$ = 0.3548   $RMSE$ = 1.598} \\ \bottomrule
      \end{tabular}
    \end{table}
    
    从表~\ref{硫酸铝钾浓度和H值指数}中可以看到,$R^2$为0.3548,且参数的置信区间跨零点,所以该模型不可用。

    \begin{table}[H]
        \centering
        \caption{硫酸铝钾浓度和S值指数增长模型回归分析结果}
        \label{硫酸铝钾浓度和S值指数}
        \begin{tabular}{@{}ccc@{}}
        \toprule
        参数        & 参数估计值      & 参数置信区间                   \\ \midrule
        $\beta_0$     & -0.01464     & {[}-0.1857, 0.1564{]}     \\
        $\beta_1$     & -0.02784    & {[}-0.0898, 0.03411{]}    \\
        \hline
        \multicolumn{3}{c}{$R^2$ = 0.551   $RMSE$ = 1.333} \\ \bottomrule
        \end{tabular}
      \end{table}
    
      从表~\ref{硫酸铝钾浓度和S值指数}中可以看到,$R^2$为0.551,仅稍好于比 H 值回归模型,且参数的置信区间跨零点,模型不可用。

    
    然后选择$Michaelis-Menten$模型对该组数据进行回归分析。取硫酸铝钾浓度作为自变量,H值和S值作为因变量,带入模型:
    
    $$y=\frac{\beta_0x}{\beta_1+x}+\beta_2$$
    硫酸铝钾浓度和H值的拟合结果如表~\ref{硫酸铝钾浓度和H值M-M}所示。
    \begin{table}[H]
      \centering
      \caption{H 值为自变量的$Michaelis-Menten$模型拟合结果}
      \label{硫酸铝钾浓度和H值M-M}
      \begin{tabular}{@{}ccc@{}}
      \toprule
      参数        & 参数估计值      & 参数置信区间                   \\ \midrule
      $\beta_0$     & 27.33     & {[}22.96, 31.7{]}     \\
      $\beta_1$     & 0.07594   & {[}-0.03905, 0.1909{]}    \\
      $\beta_2$     & 74.67     & {[}71.33, 78{]}         \\
      \hline
      \multicolumn{3}{c}{$R^2$ = 0.9941   $RMSE$ = 1.048} \\ \bottomrule
      \end{tabular}
    \end{table}
    
    从表~\ref{硫酸铝钾浓度和H值M-M}中可以看到,$R^2$接近1,$RMSE$的值为1.048,虽然参数$\beta_1$的置信区间包括零点,
    但因为$\beta_1$为常数项,不是总体参数,模型可用。

    同理,计算硫酸铝钾浓度和S值的拟合结果如表~\ref{硫酸铝钾浓度和S值M-M}所示。
    
    \begin{table}[H]
      \centering
      \caption{S 值为自变量的$Michaelis-Menten$模型拟合结果}
      \label{硫酸铝钾浓度和S值M-M}
      \begin{tabular}{@{}ccc@{}}
      \toprule
      参数        & 参数估计值      & 参数置信区间                   \\ \midrule
      $\beta_1$     & 160.1     & {[}118.6, 201.7{]}     \\
      $\beta_2$     & 0.2812   & {[}-0.02671, 0.5891{]}    \\
      $\beta_3$     & 42.5     & {[}13.38, 71.63{]}         \\
      \hline
      \multicolumn{3}{c}{$R^2$ = 0.9842   $RMSE$ = 9.159} \\ \bottomrule
      \end{tabular}
    \end{table}
    
    从表~\ref{硫酸铝钾浓度和S值M-M}中可以看到,$R^2$接近1,$RMSE$的值为9.159,相比浓度和 H 值的回归结果较差。
    选择硫酸铝钾浓度和 H 值的建立的$Michealis-Menten$模型得到硫酸铝钾颜色和浓度的模型:
<<<<<<< HEAD
    %$$y=\frac{27.33x}{0.07594+x}+74.67$$
    $$c=\frac{0.38H-28.35}{510-5H}$$
=======
    $$y=\frac{27.33x}{0.07594+x}+74.67$$
>>>>>>> ef22837dc5aaa1ff3a761eab075d00f510e8a672
    
    \subsection{奶中尿素颜色与浓度的关系}
    计算奶中尿素浓度和个颜色参数的相关性如表~\ref{奶中尿素浓度相关性}所示。
    
    \begin{table}[H]
      \centering
      \caption{奶中尿素浓度和各颜色参数的相关性}
      \label{奶中尿素浓度相关性}
      \begin{tabular}{@{}ccccccccc@{}}
      \toprule
      属性 & R  & G & B & H & S&灰度 & RGB算术平均数 & RGB几何平均数\\ \midrule
      大小 & -0.4369 & -0.3451  &-0.9603 & 0.6484 &0.9433 & -0.6165 & -0.8551 & -0.8785\\ \bottomrule
      \end{tabular}
    \end{table}
    
    观察表~\ref{奶中尿素浓度相关性}发现B的值和浓度的相关性较强而与灰度的相关性较差,
    以颜色参数B为自变量,奶中尿素浓度为因变量建立一元线性回归模型
    $$ y = p_1 \times x + p_2$$
    其中$x$为颜色参数B,$y$为奶中尿素浓度,$p_1$,$p_2$为回归系数。利用 Matlab 软件进行求解得到的回归系数估计值及置信区间,
    检验统计量 $R^2, RMSE $结果如表~\ref{奶中尿素和B值回归}所示。
    
    \begin{table}[H]
      \centering
      \caption{奶中尿素浓度和B值的回归结果}
      \label{奶中尿素和B值回归}
      \begin{tabular}{@{}ccc@{}}
      \toprule
      参数       & 参数估计值      & 参数置信区间                   \\ \midrule
      $p_1$     & -129.7     & {[}-182.1, -77.37{]}     \\
      $p_2$     & 1.549e+04   & {[}9570, 2.141e+04{]}    \\
      \hline
      \multicolumn{3}{c}{$R^2$ = 0.9221   $RMSE$ = 254.5} \\ \bottomrule
      \end{tabular}
    \end{table}
    
    从表~\ref{奶中尿素和B值回归}可以看出 $R^2$ 为0.9221,且每个回归系数的置信区间没有包含零点,说明B值和浓度的回归模型整体可用,
    得到奶中尿素浓度和颜色参数B值之间的模型:
    $$ y = -129.7x + 1.549e^{4}$$
    
    \subsection{数据质量评价}
    建立数据评价模型:
    %$$Q=W_1\gamma_1+W_2\gamma_2+W_3\gamma_3+W_4\gamma_4$$
    $$Q_i=W_1Accuracy_i+W_2Stability_i+W_3Distinction_i+W_4Match_i$$
    其中$Q$为数据质量,$W_i (i=1,2,3,4)$为权重,$Accuracy_i$、$Stability_i$、$Distinction_i$、$Match_i$分别为
    第i个物质准确度,稳定度,区分度,吻合度

    \subsubsection{准确度}
    准确度主要从实验数据的数量来考虑,按照概率论有关知识,实验测定次数越多,数据的准确度就越高,
    越有可能接近真实值。设第$i$个物质测量次数为$n_{i}$,定义准确度:
    %$$\gamma_1=\frac{n_i -\min n}{\max n-\min n}$$
    $$Accuracy_i=\frac{n_i -\min n}{\max n-\min n}$$
    
    得到表~\ref{各物质准确度}
    
    \begin{table}[H]
      \centering
      \caption{各物质准确度}
      \label{各物质准确度}
      \begin{tabular}{@{}ccc@{}}
      \toprule
      物质 & 测量组数  & 准确度 \\ \midrule
      组胺 & 10 & 0.1   \\    \bottomrule
      溴酸钾 & 10 & 0.1  \\    \bottomrule
      工业碱 & 7 & 0     \\      \bottomrule
      硫酸铝钾 & 37 & 1   \\     \bottomrule
      奶中尿素 & 15 & 0.27 \\    \bottomrule
      \end{tabular}
    \end{table}
    
    \subsubsection{稳定度}
    同种物质,在相同浓度下的R, G, B数值应该相对稳定,利用同种物质同一浓度下的变异系数衡量实验数据的稳定度。
    由于工业碱个浓度的数值只有一组,存在偶然性,计算变异系数没有意义,故将工业碱归一化的值取中为0.5。

    定义稳定度公式
    %$$\gamma_2=1-C.v$$
    $$Stability_i=1-{C.v}_i$$
    其中$C.v$表示第i个物质的变异系数,其计算公式为

    $$C.v = \frac{SD}{Me} \times 100\% = 
    \frac{\sqrt{\frac{1}{N - 1} \sum_{i=1}^{N} (X_{i} - \overline{X})^2 }} {Me} \times 100\%$$

    对变异系数的结果进行归一化处理如表~\ref{不同物质的稳定度}所示。
    
    \begin{table}[H]
    \centering
    \caption{不同物质的稳定度}
    \label{不同物质的稳定度}
    \begin{tabular}{@{}ccccccccc@{}}
    \toprule
    变异系数 & 浓度   & B    & G    & R     & H     & S     & 归一化  & 稳定度  \\ \midrule
    组胺   & 0    & 3.19\% & 0.00 \% & 0.59\%  & 3.01 \% & 2.5  \% & 0.1205
     & 0.8794 \\
         & 12.5 & 2.18\% & 0.7 \% & 0.00  \%  & 0.00    \% & 1.87\%  &      &      \\
         & 25   & 2.32\% & 0.00 \%   & 0.00 \%    & 0.00 \%    & 2.28 \% &      &      \\
         & 50   & 0.00 \%  & 0.00   \% & 0.6 \%  & 0.00  \%   & 0.92 \% &      &      \\
         & 100  & 3.93 \%& 2.18\% & 0.65\%  & 6.15\%  & 1.24 \% &      &      \\
    溴酸钾  & 0    & 0.55\% & 0.00   \% & 0.49\%  & 3.14 \% & 2.57\%  & 0    & 1    \\
         & 12.5 & 1.64\% & 0.51\% & 0.49 \% & 0 \%    & 2.72\%  &      &      \\
         & 25   & 1.02 \%& 0.52\% & 0.49\%  & 0  \%   & 0.53 \% &      &      \\
         & 50   & 3.63\% & 0.00   \% & 0.00 \%    & 0.00    \% & 2.87\%  &      &      \\
         & 100  & 0.00  \%  & 0.00   \% & 0.00   \%  & 0.00 \%    & 0.29 \% &      &      \\
    硫酸铝钾 & 0    & 1.71\% & 0.79 \%& 0.61\%  & 2.89 \% & 6.09 \% & 0.9064 & 0.0935 \\
         & 0.5  & 4.31\% & 3.01 \%& 19.26 \%& 1.15\%  & 16.22\% &      &      \\
         & 1    & 3.76 \%& 2.78 \%& 10.88 \%& 0.52\%  & 6.26 \% &      &      \\
         & 1.5  & 0.27 \%& 1.19 \%& 7.98  \%& 0.51 \% & 3.64 \% &      &      \\
         & 2    & 1.65 \%& 2.60  \%& 6.19  \%& 0.53 \% & 1.72 \% &      &      \\
         & 5    & 1.36 \%& 4.51\% & 20.64 \%& 0.89 \% & 7.45 \% &      &      \\
    奶中尿素 & 0    & 2.98\% & 0.43\% & 0.72 \% & 13.58\% & 14.7 \% & 1    & 0    \\
         & 500  & 2.63\% & 1.84 \%& 1.49 \% & 7.90  \% & 15.79\% &      &      \\
         & 1000 & 2.57 \%& 2.11\% & 2.08 \% & 10.29 \%& 17.68 \%&      &      \\
         & 1500 & 2.65\% & 0.85 \%& 0.42 \% & 2.22 \% & 7.35 \% &      &      \\
         & 2000 & 1.43\% & 1.93 \%& 1.90 \%  & 9.68 \% & 2.62  \%&      &      \\
    工业碱  & \null    & 0    & 0    & 0     & 0     & 0     & 0.5  & 0.5  \\ \bottomrule
    \end{tabular}
    \end{table}
    
    \subsubsection{区分度}
    在对同一种物质不同浓度利用比色法进行实验时,我们希望每组的读数差异相对较大,
    易于观察,用同种物质不同浓度下的变异系数离散度衡量数据的区分度。
    %$$\gamma_3=C.v$$
    $$Distinction_i={C.v}_i$$

    \begin{table}[H]
    \centering
    \caption{不同物质的区分度}
    \label{不同物质区分度}
    \begin{tabular}{@{}lllllll@{}}
    \toprule
    物质   & B     & G     & R     & H     & S     & 归一化         \\ \midrule
    组胺   & 22.97 \%& 17.78\% & 3.59 \% & 23.72 \%& 18.15 \%& 0.306382158 \\
    溴酸钾  & 59.50 \%& 2.42 \% & 1.27 \% & 6.88\%  & 55.82 \%& 0.561411402 \\
    工业碱  & 16.85 \%& 62.41\% & 12.24\% & 15.51\% & 87.12 \%& 1           \\
    硫酸铝钾 & 10.51 \%& 5.58 \% & 41.34\% & 10.11 \%& 34.91\% & 0.410759046 \\
    奶中尿素 & 5.47  \%& 1.62 \% & 1.51 \% & 7.46 \% & 22.48\% & 0           \\ \bottomrule
    \end{tabular}
    \end{table}
    
    
    \subsubsection{吻合度}
    吻合度定义为从RGB到HS的吻合度。RGB与HS分别属于两种不同的颜色判定的体系,通过RGB 和HSV 模型之间的转换的公式将每组的数据中的RGB的值
    变换到相应的HS 的值,并与附件中的HS 的值比较,计算误差。定义吻合度为
    %$$\gamma_4=(1-\frac{\sum\frac{|S-S^{'}|}{S}}{n})*100\%$$
    $$Match_i=\frac{1}{2}[(1-\frac{\sum\frac{|S^{i}_{k}-S^{i^{'}}_k|}{S^{i}_{k}}}{n})
    +(1-\frac{\sum\frac{|H^{i}_{k}-H^{i^{'}}_k|}{H^{i}_{k}}}{n})]\times 100\%$$
    分别计算H值和S值的吻合度
    \begin{table}[H]
      \centering
      \caption{不同物质的H值和S值吻合度}
      \label{吻合度表}
      \begin{tabular}{@{}ccc@{}}
      \toprule
      物质 & H值吻合度  & S值吻合度   \\ \midrule
      组胺 & 97.90\% & 99.49\% \\
      溴酸钾 & 98.74\% & 99.34\% \\
      工业碱 & 99.78\% & 98.24\% \\
      硫酸铝钾 & 99.68\% & 99.44\% \\
      奶中尿素 & 98.38\% & 99.15\% \\ \bottomrule
      \end{tabular}
    \end{table}
    
    观察表~\ref{吻合度表}发现,各组物质的吻合度都接近百分之百,实验数据的记录比较准确,
    可以简化数据评价模型为
    $$Q=W_1\gamma_1+W_2\gamma_2+W_3\gamma_3$$
    
    根据之前所计算的准确度,稳定度,区分度的个数值,设置权重均为$\frac{1}{3}$, 得到数据质量评价表\ref{数据质量表}
    
    \begin{table}[H]
      \centering
      \caption{不同物质的数据质量评价表}
      \label{数据质量表}
      \begin{tabular}{@{}ccc@{}}
      \toprule
      物质 & 数据质量  \\ \midrule
      组胺 & 0.4236  \\
      溴酸钾 & 0.5538  \\
      工业碱 & 0.4977  \\
      硫酸铝钾 & 0.5014 \\
      奶中尿素 & 0.0888 \\ \bottomrule
      \end{tabular}
    \end{table}
    
    
    观察表~\ref{数据质量表},发现除奶中尿素的数据质量较差以外,其他几种物质的数据质量都尚可。
    
    
    