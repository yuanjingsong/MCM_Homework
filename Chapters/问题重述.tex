\section{问题重述}
比色法是目前常用的一种检测物质浓度的方法,即把待测物质制备成溶液后滴在特定的白色试纸表面,
等其充分反应以后获得一张有颜色的试纸,再把该颜色试纸与一个标准比色卡进行对比,就可以确定待测物质的浓度档位。
现在我们需要给出一个精确的方法通过测量颜色读数从而获得待测物质的浓度。
\subsection{问题一}
(1)通过附件1所给出的5组数据确定各物质颜色读数和物质浓度的关系

(2)给出评价标准并评价一直数据的精准程度

\subsection{问题二}
(1)通过附件2所给出的模型,建立颜色读数和浓度间的模型

(2)通过(1)中建立的模型进行误差分析

\subsection{问题三}
(1)探讨数据量对模型的影响

(2)探讨颜色维度对模型的影响

\section{问题假设}
假设R,G,B,H,S的数据都是实际测量未受主管影响的
测量的物质中杂质或其他成分对颜色没有影响

\section{符号说明}
