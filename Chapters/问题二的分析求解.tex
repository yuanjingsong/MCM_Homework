\section{问题二的模型建立与求解}

\subsection{数据处理及分析}

利用问题一中建立的数据质量综合评价模型,综合评价问题一中五种物质以及二氧化硫的数据质量。
首先计算有关二氧化硫的测量数据中,RGB值与HS值的吻合度,H、S值的测量值与H、S的计算值如表~\ref{SO2_HSCheck}所示。

\begin{table}[]
    \centering
    \caption{二氧化硫数据表中H、S的测量值与计算值(只取前10项)}
    \label{SO2_HSCheck}
    \begin{tabular}{@{}llll@{}}
    \toprule
    S记录值 & H记录值 & H计算值        & S计算值        \\ \midrule
    138  & 14   & 136.67 & 14.62 \\
    138  & 16   & 138         & 16.24 \\
    137  & 20   & 137.5       & 19.37 \\
    137  & 20   & 137.5       & 19.37 \\
    141  & 19   & 142.5       & 19.49 \\
    135  & 82   & 135.81 & 82.5        \\
    136  & 81   & 136.11 & 81.48 \\
    135  & 83   & 135.79 & 84.51 \\
    135  & 87   & 135.5       & 87.93 \\
    135  & 89   & 135         & 89.83 \\ \bottomrule
    \end{tabular}
    \end{table}

由表~\ref{SO2_HSCheck}可以发现,H、S的测量值分别与S、H的计算值高度吻合,由此推测是记录数据时记录人员弄反了H、S这两列。
在原表格中将其修正,计算其吻合度如表~\ref{SO2_HSMatch}所示。可见RGB与HS也基本高度吻合,符合使用问题一中数据质量综合评价模型的条件。

\begin{table}[]
    \centering
    \caption{二氧化硫H、S校正后地值与计算值地吻合度}
    \label{SO2_HSMatch}
    \begin{tabular}{@{}ll@{}}
    \toprule
    参数 & 吻合度     \\ \midrule
    H  & 99.62\% \\
    S  & 98.83\% \\ \bottomrule
    \end{tabular}
    \end{table}

重新计算各项目的准确度、稳定度、区分度,得到数据质量的量化评价如表~\ref{SO2_Judge}所示。

\begin{table}[]
    \centering
    \caption{问题一中五种物质与二氧化硫的数据质量综合评价结果}
    \label{SO2_Judge}
    \begin{tabular}{@{}lllll@{}}
    \toprule
    项目   & 准确度  & 稳定度  & 区分度  & 数据质量 \\ \midrule
    组胺   & 0.10 & 0.88 & 0.31 & 0.43 \\
    溴酸钾  & 0.10 & 1.00 & 0.56 & 0.55 \\
    工业碱  & 0.00 & 0.59 & 1.00 & 0.53 \\
    硫酸铝钾 & 1.00 & 0.09 & 0.41 & 0.50 \\
    奶中尿素 & 0.27 & 0.00 & 0.00 & 0.09 \\
    二氧化硫 & 0.60 & 1.00 & 0.22 & 0.61 \\ \bottomrule
    \end{tabular}
    \end{table}

由表~\ref{SO2_Judge}可见,与二氧化硫有关的测量数据质量较好,适合进行研究分析。

比较各参数与浓度的线性相关性,结果如表~所示。

(表格:二氧化硫各系数与其浓度的线性相关系数)

可以发现灰度和G与浓度c有较强的线性相关性,其折线图如图~所示,它们都与浓度呈负相关。

(图:二氧化硫浓度关于灰度与G的折线图)

\subsection{一元线性回归模型}

根据数据分析的结果,首先分别建立基于灰度(Gr)与G的一元线性回归模型。

对于灰度,建立一元线性回归方程:

$$c=a \times Gr+b$$

求解结果如表~所示。

(表格:二氧化硫浓度关于灰度的一元线性回归结果)

对于G,建立一元线性回归方程:

$$c=a \times G+b$$

求解结果如表~所示。

(表格:二氧化硫浓度关于G的一元线性回归结果)

由表~和表~所示,这两种一元线性回归模型的回归结果较不理想,R方值过低,无法应用于实际。

\subsection{指数增长模型}

观察浓度与灰度的关系,如图~所示,发现浓度随灰度的变化趋势近似符合指数模型。
设: $$c=ae^{b\dot Gr}$$,利用Matlab的拟合工具箱计算得出结果如表~所示。
由表中结果可以看出,虽然该模型的R方值较小,但是参数a的置信区间跨越了0点,该模型无法被采纳应用于实际。

(图片:浓度随灰度的变化趋势)

(表格:二氧化硫浓度关于灰度的指数增长回归结果)



\subsection{快速平衡模型}

在问题一中,快速平衡模型很好的描述了硫酸铝钾的H、S与其浓度的关系。在本问题中继续就此模型展开讨论。
绘制二氧化硫H、S(修正后)的值随其浓度变化的折线图,观察其趋势,如图~所示。
从图~中可以看出,H与浓度没有明显的关系,而S的变化趋势则符合使用快速平衡模型的条件。

(图:二氧化硫H、S随其浓度c变化的折线图)

建立回归模型:


   $$ S = \frac{\beta_1 c}{\beta_2b+\beta_3}$$


求解结果如表~所示,该模型拟合效果较好,且参数的置信区间没有包含零点,说明模型是可信的。
将模型转化为浓度关于S的表达式:


    $$c = ...$$


综合以上的讨论,基于S的快速平衡模型能较好地解决此问题。

(表:二氧化硫S与浓度的快速平衡模型计算结果)
